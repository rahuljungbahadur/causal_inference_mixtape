% Options for packages loaded elsewhere
\PassOptionsToPackage{unicode}{hyperref}
\PassOptionsToPackage{hyphens}{url}
\PassOptionsToPackage{dvipsnames,svgnames,x11names}{xcolor}
%
\documentclass[
  letterpaper,
  DIV=11,
  numbers=noendperiod]{scrartcl}

\usepackage{amsmath,amssymb}
\usepackage{lmodern}
\usepackage{iftex}
\ifPDFTeX
  \usepackage[T1]{fontenc}
  \usepackage[utf8]{inputenc}
  \usepackage{textcomp} % provide euro and other symbols
\else % if luatex or xetex
  \usepackage{unicode-math}
  \defaultfontfeatures{Scale=MatchLowercase}
  \defaultfontfeatures[\rmfamily]{Ligatures=TeX,Scale=1}
\fi
% Use upquote if available, for straight quotes in verbatim environments
\IfFileExists{upquote.sty}{\usepackage{upquote}}{}
\IfFileExists{microtype.sty}{% use microtype if available
  \usepackage[]{microtype}
  \UseMicrotypeSet[protrusion]{basicmath} % disable protrusion for tt fonts
}{}
\makeatletter
\@ifundefined{KOMAClassName}{% if non-KOMA class
  \IfFileExists{parskip.sty}{%
    \usepackage{parskip}
  }{% else
    \setlength{\parindent}{0pt}
    \setlength{\parskip}{6pt plus 2pt minus 1pt}}
}{% if KOMA class
  \KOMAoptions{parskip=half}}
\makeatother
\usepackage{xcolor}
\setlength{\emergencystretch}{3em} % prevent overfull lines
\setcounter{secnumdepth}{-\maxdimen} % remove section numbering
% Make \paragraph and \subparagraph free-standing
\ifx\paragraph\undefined\else
  \let\oldparagraph\paragraph
  \renewcommand{\paragraph}[1]{\oldparagraph{#1}\mbox{}}
\fi
\ifx\subparagraph\undefined\else
  \let\oldsubparagraph\subparagraph
  \renewcommand{\subparagraph}[1]{\oldsubparagraph{#1}\mbox{}}
\fi

\usepackage{color}
\usepackage{fancyvrb}
\newcommand{\VerbBar}{|}
\newcommand{\VERB}{\Verb[commandchars=\\\{\}]}
\DefineVerbatimEnvironment{Highlighting}{Verbatim}{commandchars=\\\{\}}
% Add ',fontsize=\small' for more characters per line
\usepackage{framed}
\definecolor{shadecolor}{RGB}{241,243,245}
\newenvironment{Shaded}{\begin{snugshade}}{\end{snugshade}}
\newcommand{\AlertTok}[1]{\textcolor[rgb]{0.68,0.00,0.00}{#1}}
\newcommand{\AnnotationTok}[1]{\textcolor[rgb]{0.37,0.37,0.37}{#1}}
\newcommand{\AttributeTok}[1]{\textcolor[rgb]{0.40,0.45,0.13}{#1}}
\newcommand{\BaseNTok}[1]{\textcolor[rgb]{0.68,0.00,0.00}{#1}}
\newcommand{\BuiltInTok}[1]{\textcolor[rgb]{0.00,0.23,0.31}{#1}}
\newcommand{\CharTok}[1]{\textcolor[rgb]{0.13,0.47,0.30}{#1}}
\newcommand{\CommentTok}[1]{\textcolor[rgb]{0.37,0.37,0.37}{#1}}
\newcommand{\CommentVarTok}[1]{\textcolor[rgb]{0.37,0.37,0.37}{\textit{#1}}}
\newcommand{\ConstantTok}[1]{\textcolor[rgb]{0.56,0.35,0.01}{#1}}
\newcommand{\ControlFlowTok}[1]{\textcolor[rgb]{0.00,0.23,0.31}{#1}}
\newcommand{\DataTypeTok}[1]{\textcolor[rgb]{0.68,0.00,0.00}{#1}}
\newcommand{\DecValTok}[1]{\textcolor[rgb]{0.68,0.00,0.00}{#1}}
\newcommand{\DocumentationTok}[1]{\textcolor[rgb]{0.37,0.37,0.37}{\textit{#1}}}
\newcommand{\ErrorTok}[1]{\textcolor[rgb]{0.68,0.00,0.00}{#1}}
\newcommand{\ExtensionTok}[1]{\textcolor[rgb]{0.00,0.23,0.31}{#1}}
\newcommand{\FloatTok}[1]{\textcolor[rgb]{0.68,0.00,0.00}{#1}}
\newcommand{\FunctionTok}[1]{\textcolor[rgb]{0.28,0.35,0.67}{#1}}
\newcommand{\ImportTok}[1]{\textcolor[rgb]{0.00,0.46,0.62}{#1}}
\newcommand{\InformationTok}[1]{\textcolor[rgb]{0.37,0.37,0.37}{#1}}
\newcommand{\KeywordTok}[1]{\textcolor[rgb]{0.00,0.23,0.31}{#1}}
\newcommand{\NormalTok}[1]{\textcolor[rgb]{0.00,0.23,0.31}{#1}}
\newcommand{\OperatorTok}[1]{\textcolor[rgb]{0.37,0.37,0.37}{#1}}
\newcommand{\OtherTok}[1]{\textcolor[rgb]{0.00,0.23,0.31}{#1}}
\newcommand{\PreprocessorTok}[1]{\textcolor[rgb]{0.68,0.00,0.00}{#1}}
\newcommand{\RegionMarkerTok}[1]{\textcolor[rgb]{0.00,0.23,0.31}{#1}}
\newcommand{\SpecialCharTok}[1]{\textcolor[rgb]{0.37,0.37,0.37}{#1}}
\newcommand{\SpecialStringTok}[1]{\textcolor[rgb]{0.13,0.47,0.30}{#1}}
\newcommand{\StringTok}[1]{\textcolor[rgb]{0.13,0.47,0.30}{#1}}
\newcommand{\VariableTok}[1]{\textcolor[rgb]{0.07,0.07,0.07}{#1}}
\newcommand{\VerbatimStringTok}[1]{\textcolor[rgb]{0.13,0.47,0.30}{#1}}
\newcommand{\WarningTok}[1]{\textcolor[rgb]{0.37,0.37,0.37}{\textit{#1}}}

\providecommand{\tightlist}{%
  \setlength{\itemsep}{0pt}\setlength{\parskip}{0pt}}\usepackage{longtable,booktabs,array}
\usepackage{calc} % for calculating minipage widths
% Correct order of tables after \paragraph or \subparagraph
\usepackage{etoolbox}
\makeatletter
\patchcmd\longtable{\par}{\if@noskipsec\mbox{}\fi\par}{}{}
\makeatother
% Allow footnotes in longtable head/foot
\IfFileExists{footnotehyper.sty}{\usepackage{footnotehyper}}{\usepackage{footnote}}
\makesavenoteenv{longtable}
\usepackage{graphicx}
\makeatletter
\def\maxwidth{\ifdim\Gin@nat@width>\linewidth\linewidth\else\Gin@nat@width\fi}
\def\maxheight{\ifdim\Gin@nat@height>\textheight\textheight\else\Gin@nat@height\fi}
\makeatother
% Scale images if necessary, so that they will not overflow the page
% margins by default, and it is still possible to overwrite the defaults
% using explicit options in \includegraphics[width, height, ...]{}
\setkeys{Gin}{width=\maxwidth,height=\maxheight,keepaspectratio}
% Set default figure placement to htbp
\makeatletter
\def\fps@figure{htbp}
\makeatother

\KOMAoption{captions}{tableheading}
\makeatletter
\makeatother
\makeatletter
\makeatother
\makeatletter
\@ifpackageloaded{caption}{}{\usepackage{caption}}
\AtBeginDocument{%
\ifdefined\contentsname
  \renewcommand*\contentsname{Table of contents}
\else
  \newcommand\contentsname{Table of contents}
\fi
\ifdefined\listfigurename
  \renewcommand*\listfigurename{List of Figures}
\else
  \newcommand\listfigurename{List of Figures}
\fi
\ifdefined\listtablename
  \renewcommand*\listtablename{List of Tables}
\else
  \newcommand\listtablename{List of Tables}
\fi
\ifdefined\figurename
  \renewcommand*\figurename{Figure}
\else
  \newcommand\figurename{Figure}
\fi
\ifdefined\tablename
  \renewcommand*\tablename{Table}
\else
  \newcommand\tablename{Table}
\fi
}
\@ifpackageloaded{float}{}{\usepackage{float}}
\floatstyle{ruled}
\@ifundefined{c@chapter}{\newfloat{codelisting}{h}{lop}}{\newfloat{codelisting}{h}{lop}[chapter]}
\floatname{codelisting}{Listing}
\newcommand*\listoflistings{\listof{codelisting}{List of Listings}}
\makeatother
\makeatletter
\@ifpackageloaded{caption}{}{\usepackage{caption}}
\@ifpackageloaded{subcaption}{}{\usepackage{subcaption}}
\makeatother
\makeatletter
\@ifpackageloaded{tcolorbox}{}{\usepackage[many]{tcolorbox}}
\makeatother
\makeatletter
\@ifundefined{shadecolor}{\definecolor{shadecolor}{rgb}{.97, .97, .97}}
\makeatother
\makeatletter
\makeatother
\ifLuaTeX
  \usepackage{selnolig}  % disable illegal ligatures
\fi
\IfFileExists{bookmark.sty}{\usepackage{bookmark}}{\usepackage{hyperref}}
\IfFileExists{xurl.sty}{\usepackage{xurl}}{} % add URL line breaks if available
\urlstyle{same} % disable monospaced font for URLs
\hypersetup{
  pdftitle={Matching and Sub-Classification},
  pdfauthor={Rahul bahadur},
  colorlinks=true,
  linkcolor={blue},
  filecolor={Maroon},
  citecolor={Blue},
  urlcolor={Blue},
  pdfcreator={LaTeX via pandoc}}

\title{Matching and Sub-Classification}
\author{Rahul bahadur}
\date{}

\begin{document}
\maketitle
\ifdefined\Shaded\renewenvironment{Shaded}{\begin{tcolorbox}[boxrule=0pt, interior hidden, borderline west={3pt}{0pt}{shadecolor}, breakable, sharp corners, enhanced, frame hidden]}{\end{tcolorbox}}\fi

\hypertarget{bias-correction}{%
\section{Bias Correction:}\label{bias-correction}}

Causal inference relies on finding the identical twin that was exposed
to the treatment for the one which was not or vice-versa. However, that
is not always possible. If the covariates are not exactly the same then
matching them would lead to biases. This bias can be corrected as shown
below.

\begin{Shaded}
\begin{Highlighting}[]
\ImportTok{import}\NormalTok{ numpy }\ImportTok{as}\NormalTok{ np }
\ImportTok{import}\NormalTok{ pandas }\ImportTok{as}\NormalTok{ pd }
\ImportTok{import}\NormalTok{ statsmodels.api }\ImportTok{as}\NormalTok{ sm }
\ImportTok{import}\NormalTok{ statsmodels.formula.api }\ImportTok{as}\NormalTok{ smf }
\ImportTok{from}\NormalTok{ itertools }\ImportTok{import}\NormalTok{ combinations }
\ImportTok{import}\NormalTok{ plotnine }\ImportTok{as}\NormalTok{ p}
\ImportTok{import}\NormalTok{ ssl}
\end{Highlighting}
\end{Shaded}

\begin{Shaded}
\begin{Highlighting}[]
\CommentTok{\# read data}
\NormalTok{ssl.\_create\_default\_https\_context }\OperatorTok{=}\NormalTok{ ssl.\_create\_unverified\_context}
\KeywordTok{def}\NormalTok{ read\_data(}\BuiltInTok{file}\NormalTok{): }
    \ControlFlowTok{return}\NormalTok{ pd.read\_stata(}\StringTok{"https://raw.github.com/scunning1975/mixtape/master/"} \OperatorTok{+} \BuiltInTok{file}\NormalTok{)}

\NormalTok{training\_bias\_reduction }\OperatorTok{=}\NormalTok{ read\_data(}\StringTok{"training\_bias\_reduction.dta"}\NormalTok{) }
\end{Highlighting}
\end{Shaded}

\begin{Shaded}
\begin{Highlighting}[]
\NormalTok{training\_bias\_reduction}
\end{Highlighting}
\end{Shaded}

\begin{tabular}{lrrrr}
\toprule
{} &  Unit &   Y &  D &   X \\
\midrule
0 &     1 &   5 &  1 &  11 \\
1 &     2 &   2 &  1 &   7 \\
2 &     3 &  10 &  1 &   5 \\
3 &     4 &   6 &  1 &   3 \\
4 &     5 &   4 &  0 &  10 \\
5 &     6 &   0 &  0 &   8 \\
6 &     7 &   5 &  0 &   4 \\
7 &     8 &   1 &  0 &   1 \\
\bottomrule
\end{tabular}

\hypertarget{steps-for-bias-correction}{%
\subsection{Steps for Bias correction}\label{steps-for-bias-correction}}

\begin{verbatim}
1. Find the closest matching unit with the treatment unit based on the covariate (IV). For eg. for `Unit` 1 with X=11, the matching un-treated unit is X=10, i.e. `Unit` 5. Similarly for others
\end{verbatim}

\begin{Shaded}
\begin{Highlighting}[]
\NormalTok{training\_bias\_reduction[}\StringTok{\textquotesingle{}Y1\textquotesingle{}}\NormalTok{] }\OperatorTok{=}\NormalTok{ np.where(training\_bias\_reduction.D}\OperatorTok{==}\DecValTok{1}\NormalTok{, training\_bias\_reduction.Y, }\DecValTok{0}\NormalTok{)}
\NormalTok{training\_bias\_reduction[}\StringTok{\textquotesingle{}Y0\textquotesingle{}}\NormalTok{] }\OperatorTok{=}\NormalTok{ np.where(training\_bias\_reduction.D}\OperatorTok{==}\DecValTok{0}\NormalTok{, training\_bias\_reduction.Y, }\DecValTok{0}\NormalTok{)}
\end{Highlighting}
\end{Shaded}

\begin{verbatim}
2. Create a column with the fitted data using a model for `Y ~ X`
\end{verbatim}

\begin{Shaded}
\begin{Highlighting}[]
\NormalTok{fitted\_model }\OperatorTok{=}\NormalTok{ sm.OLS.from\_formula(}\StringTok{\textquotesingle{}Y \textasciitilde{} X\textquotesingle{}}\NormalTok{, training\_bias\_reduction).fit()}
\NormalTok{training\_bias\_reduction[}\StringTok{\textquotesingle{}fitted\textquotesingle{}}\NormalTok{] }\OperatorTok{=}\NormalTok{  fitted\_model.predict(training\_bias\_reduction.X)}
\NormalTok{training\_bias\_reduction}
\end{Highlighting}
\end{Shaded}

\begin{tabular}{lrrrrrrr}
\toprule
{} &  Unit &   Y &  D &   X &  Y1 &  Y0 &    fitted \\
\midrule
0 &     1 &   5 &  1 &  11 &   5 &   0 &  3.888071 \\
1 &     2 &   2 &  1 &   7 &   2 &   0 &  4.082474 \\
2 &     3 &  10 &  1 &   5 &  10 &   0 &  4.179676 \\
3 &     4 &   6 &  1 &   3 &   6 &   0 &  4.276878 \\
4 &     5 &   4 &  0 &  10 &   0 &   4 &  3.936672 \\
5 &     6 &   0 &  0 &   8 &   0 &   0 &  4.033873 \\
6 &     7 &   5 &  0 &   4 &   0 &   5 &  4.228277 \\
7 &     8 &   1 &  0 &   1 &   0 &   1 &  4.374080 \\
\bottomrule
\end{tabular}

Bias := it is the diff in the predicted values generated based on the
covariates.` implying that, given the same model, and the same set of
covariates, and no information on which units

are treated or not, the model should generate fitted values consistent
with these assumptions.

Bias reduction method := It is the diff between the diff of
\texttt{Treated} and \texttt{Un-Treated} covariate and the diff between
\texttt{Treated} predicted value and \texttt{un\_treated} predicted
value

\begin{Shaded}
\begin{Highlighting}[]
\NormalTok{ATT }\OperatorTok{=}\NormalTok{ np.mean((np.array(training\_bias\_reduction[}\StringTok{\textquotesingle{}Y\textquotesingle{}}\NormalTok{][training\_bias\_reduction.D}\OperatorTok{==}\DecValTok{1}\NormalTok{]) }\OperatorTok{{-}}\NormalTok{ np.array(training\_bias\_reduction[}\StringTok{\textquotesingle{}Y\textquotesingle{}}\NormalTok{][training\_bias\_reduction.D}\OperatorTok{==}\DecValTok{0}\NormalTok{])) }\OperatorTok{{-}}
\NormalTok{ (np.array(training\_bias\_reduction[}\StringTok{\textquotesingle{}fitted\textquotesingle{}}\NormalTok{][training\_bias\_reduction.D}\OperatorTok{==}\DecValTok{1}\NormalTok{]) }\OperatorTok{{-}}\NormalTok{ np.array(training\_bias\_reduction[}\StringTok{\textquotesingle{}fitted\textquotesingle{}}\NormalTok{][training\_bias\_reduction.D}\OperatorTok{==}\DecValTok{0}\NormalTok{])))}

\BuiltInTok{print}\NormalTok{(ATT)}
\end{Highlighting}
\end{Shaded}

\begin{verbatim}
3.2864506627393233
\end{verbatim}

\hypertarget{computing-the-variance-of-the-bias-estimator}{%
\subsection{Computing the variance of the bias
estimator}\label{computing-the-variance-of-the-bias-estimator}}

\[
\sigma^2_{ATT} = \frac{1}{N_T} \sum_{D_i=1}(Y_i - \frac{1}{M} \sum_{M_i=1}^{M} {Y_{j_{(m)}i} - \hat{\delta}_{ATT}})^2
\]

\begin{Shaded}
\begin{Highlighting}[]
\NormalTok{var\_att }\OperatorTok{=}\NormalTok{ (((np.array(training\_bias\_reduction.Y[training\_bias\_reduction.D}\OperatorTok{==}\DecValTok{1}\NormalTok{]) }\OperatorTok{{-}}\NormalTok{ np.array(training\_bias\_reduction.Y[training\_bias\_reduction.D}\OperatorTok{==}\DecValTok{0}\NormalTok{]) }\OperatorTok{{-}}\NormalTok{ ATT)) }\OperatorTok{**}\DecValTok{2}\NormalTok{).mean()}
\NormalTok{var\_att, np.sqrt(var\_att)}
\end{Highlighting}
\end{Shaded}

\begin{verbatim}
(3.188828650814136, 1.7857291650231106)
\end{verbatim}



\end{document}
